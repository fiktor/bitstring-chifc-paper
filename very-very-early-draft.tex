\documentclass{article}
% General document formatting
\usepackage[margin=0.7in]{geometry}
\usepackage[parfill]{parskip}
% Documentation:
% https://ftp.cc.uoc.gr/mirrors/CTAN/macros/latex/contrib/parskip/parskip.pdf
\usepackage[utf8]{inputenc}

% Math packages:
\usepackage{amsmath,amssymb,amsfonts,amsthm}

% Our definitions:
\DeclareMathOperator{\smoothen}{smoothen}
\DeclareMathOperator{\mean}{mean}

\begin{document}
\section{Introduction}
TODO:

Such a family can arise, e.g., from measurements of a low-temperature
Gibbs ensemble of Hamiltonians parametrized by a parameter $\lambda$.

\section{Classical Fidelity Susceptibility}
Classical fidelity between 2 probability distributions $p$ and $q$ of
bitstrings $z$ is defined as
\begin{equation}
  F_c(p, q) = \sum_{z} \sqrt{p(z) q(z)}.
\end{equation}
We are interested in the fidelity between bitstring distributions at different
$s$ (e.g. $s=s_1$ and $s=s_2$), which we will denote as $F_c(s_1, s_2)$.

Fidelity susceptibility is defined as the term $\chi_{F_c}(s)$ in the Taylor
expansion
\begin{equation}
\label{eq:Fcs.Tailor}
  F_c(s, s+\delta s) = 1 - \frac{\delta s^2}{2} \chi_{F_c}(s) + O(\delta s^3).
\end{equation}
For such Taylor expansion to exist it is sufficient that the probabilities have
a Taylor expansion up to $O(\delta s^3)$. More generally, probability
distribution can depend on a point $\lambda$ on a manifold $\Lambda$,
in which case the Tailor expansion \eqref{eq:Fcs.Tailor} would become
\begin{equation}
\label{eq:Fcl.Tailor}
  F_c(\lambda, \lambda+\delta \lambda) = 1 - \frac{\delta \lambda_{j} \delta \lambda_k}{2} \chi_{F_c}^{jk}(\lambda) + O(\delta \lambda^3).
\end{equation}

\subsection{Classical and quantum fidelity susceptibility}
Fact 1: For pure states $\mathbb{E}\chi_{F_c} (s) = \frac12 \chi_F (s)$
where the expectation is over all orthogonal bases to perform the measurement in.

TODO:proof

Fact 2: For computational basis measurement of a non-degenerate ground state
of a real-valued Hamiltonian $H$, then $\chi_{F_c}(s) = \chi_F(s)$
almost everywhere.

TODO:proof

\section{Problem setup}
\begin{itemize}
  \item In this work we consider a family of distributions of
bitstrings $\{\mathcal{D}_{\lambda}\}_{\lambda \in \Lambda}$, each of length $n$.
  \item We are given a finite sample $\mathcal{D}_{\textrm{train}}$ of size $N$
  of pairs $(\lambda, z)$ s.t. $P(z|\lambda) = P_{\mathcal{D}_{\lambda}}(z)$.
  \item We are also given (possibly implicitly via coordinate description of $\Lambda$) a naive metric $g^0$ on $\Lambda$.
  \item We are asked to estimate the Fisher information metric $g$ on $\Lambda$ corresponding to distributions $\mathcal{D}_{\lambda}$.
  \item Locations with high $g / g^0$ are then considered to be conjectured locations of possible phase transitions.
\end{itemize}

We focus on the task of identifying phase transitions in that
family. Rigorously speaking, phase transitions are only defined
in the limit $n\to\infty$, while we are dealing with finite size systems.
A solution to that is to look at Fisher information metric: high distances according to Fisher information metric for points close according to naive metric likely correspond to phase transitions.

\section{Bitstring-ChiFc method}
In this work we propose the following method:
\begin{itemize}
  \item Collect training dataset $\mathcal{D}_{\chi_{F_c}\textrm{-train}}$ of
  the form $(\lambda_0, \delta \lambda, z, y)$, where $z$ is sampled from
  $p(\bullet, \lambda=\lambda_{z})$,
  $p_{+} = p(\lambda_{z} = \lambda_0 + \delta \lambda / 2| \lambda_{z} = \lambda_0 \pm \delta \lambda / 2)$, and
    $\mathbb{E}(y|\lambda_0, \delta \lambda, z) = p_{+}$.
    In practice $y \in \{0, 1\}$.
    Do it in the following way:
    \begin{itemize}
      \item Consider $\mathcal{D}_{\textrm{train}}$ consisting of pairs
      $(z, \lambda)$.
      \item Sample pairs $(z_{i{+}}, \lambda_{i{+}})$,
      $(z_{i{-}}, \lambda_{i{-}})$ from $\mathcal{D}_{\textrm{train}}$.
      \item Compute $\lambda_i = (\lambda_{i{+}} + \lambda_{i{-}})/2$,
      $\delta \lambda_i = \lambda_{i{+}} - \lambda_{i{-}}$.
      \item Add tuples $(\lambda_i, \delta \lambda_i, z_{i{+}}, 1)$ and $(\lambda_i, \delta \lambda_i, z_{i{-}}, 0)$ to the dataset $\mathcal{D}_{\chi_{F_c}\textrm{-train}}$.
    \end{itemize}
  \item Train a model $M$, which given $(\lambda_0, \delta \lambda, z)$
  will predict $l = M(\lambda_0, \delta \lambda, z)$
  s.t. $p_{+} = (1+e^{-l \cdot \delta \lambda})^{-1}$.
  Do this by minimizing cross-entropy loss on the dataset
  $\mathcal{D}_{\chi_{F_c}\textrm{-train}}$.
  \item Estimate
  \begin{equation}
  \chi_{F_c}^{jk}(\lambda) = \smoothen\left(\lambda_1 \mapsto
  \mean_{(z, \lambda_1) \in \mathcal{D}_{\textrm{train}}}
  M(\lambda_1, 0, z)^{j}M(\lambda_1, 0, z)^{k}\right)(\lambda).
  \end{equation}
  \item Derivation:
    \begin{multline*}
      \chi_{F_c}(\lambda) = \lim_{\delta \lambda\to 0} \frac2{\delta \lambda^2} \left(1 - \mathbb{E}_{z\sim Q(\bullet)} \frac{\sqrt{P(z|\lambda-\delta \lambda/2)P(z|\lambda+\delta \lambda/2)}}{Q(z)}\right)
      \\ \simeq \lim_{\delta \lambda\to 0} \mathbb{E}_{Q}\frac2{\delta \lambda^2}\frac{2\sinh^2(l\delta \lambda/4)}{\cosh(l\delta \lambda/2)} \simeq \frac14 \mathbb{E}_{z|\lambda} M(\lambda, 0, z)^2.
    \end{multline*}
\end{itemize}


\end{document}
