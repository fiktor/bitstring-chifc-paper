\documentclass[twocolumn]{revtex4-2}
\usepackage{hyperref}
\usepackage{amsmath}
\usepackage{tikz}
\usepackage{braket}

% Our definitions:
%\DeclareMathOperator{\Tr}{Tr}
\newcommand{\abs}[1]{\left|#1\right|}

\newtheorem{theorem}{Theorem}
\newtheorem{lemma}[theorem]{Lemma}

% Draft comments
\usepackage{xcolor}
\newcommand{\VK}[1]{\textcolor{teal}{[VK: #1]}}
\newcommand{\DL}[1]{\textcolor{blue}{[DL: #1]}}
\newcommand{\NE}[1]{\textcolor{magenta}{[NE: #1]}}

\begin{document}

\title{Machine learning Fisher Information Metric from bitstrings}
\author{First Last}
\email{email}
\affiliation{USC affiliation}

\date{\today}

\bibliography{biblio.bib}
\bibliographystyle{apsrev4-1}

\maketitle

\section{Introduction}
\subsection{Observation}
Ideal quantum computers are conjectured to exhibit
a quantum speedup in returning samples from
low-energy states of $H$.

\section{Caveats}
\subsection{Quantum speedup in sampling}
Consider a Hamiltonian. For example, it can be picked from one of the
following families.
\begin{itemize}
  \item \textbf{Lattice Hamiltonian.} A Hamiltonian on a $d$-dimensional
    rectangular grid:
    \begin{multline}
      H = \sum_{i} (h^{X}_{i} X_i + h^{Z}_i Z_i)
        + \sum_{\left<i,j\right>} J^{XZ}_{ij}X_iZ_j \\
          + \frac12\sum_{\left<i,j\right>}\Bigl(J^{XX}_{ij}X_iX_j
          + J^{YY}_{ij}Y_iY_j + J^{ZZ}_{ij}Z_iZ_j\Bigr).
    \end{multline}
    Here $\left<i,j\right>$ in the sum indicates that the sum is over
    the pairs of nodes $i$ and $j$ connected by an edge,
    $J^{XX}_{ij} = J^{XX}_{ji}$,
    $J^{YY}_{ij} = J^{YY}_{ji}$,
    $J^{ZZ}_{ij} = J^{ZZ}_{ji}$.
    We picked specifically the terms $X$, $Z$, $XX$, $XZ$, $YY$, $ZZ$ to
    keep the elements of the Hamiltonian in the computational basis real,
    which is desireable due to theorem \VK{TODO:ref} below.
  \item \textbf{Hamiltonian on a random graph}
    A Hamiltonian on a graph (e.g. a MAXCUT Hamiltonian
    on a 3-regular graph).
\end{itemize}
While Hamiltonian on a random graph presents a challenge to most of the
classical algorithms, the problem of finding the ground state of such
Hamiltonian was not studied extensively in the literature.

We define the sampling problem as follows. Let $\ket{\psi_0}$ be the ground
state of the Hamiltonian and $p$ be the corresponding probability distribution
with $p_z = \abs{\braket{z|\psi_0}}^2$.

\textbf{Question.} Would we expect a quantum algorithm to have a significant
advantage over classical algorithms in this problem?

Here are the results related to this question.

Do quantum computers exhibit a quantum speedup in finding low
energy states?

Classically the ground state of 1-dimensional systems can be approximated using
DMRG in polynomial time.

\end{document}
